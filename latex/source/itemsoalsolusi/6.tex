\item
Piston dalam mesin berosilasi seperti gerak harmonis sederhana. Posisi berubah-ubah berdasarkan persamaan:
\begin{eqnarray*}
x=(5.0 \mbox{ cm})\cos(2t+\frac{\pi}{6})
\end{eqnarray*}
Pada $t=0$ s , tentukan: \\
a. Posisi piston \\
b. kecepatan piston \\
c. percepatan piston \\
d. Periode dan amplitudo
\begin{description}
    \item[Solusi:]
a. pada $t=0$
\begin{eqnarray*}
x&=&(5.0 \mbox{ cm})\cos(2(0)+\frac{\pi}{6}) \\
&=&(5.0 \mbox{ cm})\cos(\frac{\pi}{6}) \\
&=&4.33 \mbox{ cm}
\end{eqnarray*}
b. 
\begin{eqnarray*}
v&=&\frac{dx}{dt}=-(2)(5.0)\sin(2t+\frac{\pi}{6}) \quad \mbox{pada $t=0$, maka} \\
v&=&-10.0 \sin(\frac{\pi}{6}) \\
v&=&-5.0 \mbox{ cm/s}
\end{eqnarray*}
c.
\begin{eqnarray*}
a&=&\frac{dv}{dt}=-(2)(10) \cos(2t+\frac{\pi}{6}) \quad \mbox{pada $t=0$, maka} \\
a&=&-20.0 \cos(\frac{\pi}{6}) \\
a&=&-17.3 \mbox{ $cm/s^2$}
\end{eqnarray*}
d.\quad Amplitudo $=$ 5.0 cm \\
\begin{eqnarray*}
\textrm{periode =}T=\frac{2\pi}{\omega}=\frac{2\pi}{2}=\pi=3.14 \mbox{ s}
\end{eqnarray*}
\\[1.5cm]

\end{description}
