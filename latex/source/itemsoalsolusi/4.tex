\item
Mobil bergerak dengan kelajuan 1.0 m/s. Jika koefisien gesek kinetik antara ban dengan jalan sebesar 0.1\\
Ketika mobil menginjak rem, berapa jarak minimum mobil sampai benar-benar berhenti?
\begin{description}
    \item[Solusi:]
Gaya yang dialami saat terjadi perlambatan,
\begin{eqnarray*}
F&=&-\mu_{s}mg \quad \mbox{dari hukum newton II $F=ma$} \\
ma&=&-\mu_{s}mg \\
a&=&-\mu_{s}g \\
&=&-(0.1)(9.8) \\
&=&-0.98 \textrm{ $m/s^2$}
\end{eqnarray*}

Karena mobil mengalami perlambatan konstan, maka berlaku
\begin{eqnarray*}
v_{f}^{2}-v_{i}^{2}&=&2as  \quad \mbox{karena mobil mengerem jadi $v_{f}=0$} \\
-v_{i}^{2}&=&2as \\
-1.0^2&=&2(-0.98)s \\
s&=&0.5102 \textrm{ m}
\end{eqnarray*}
\end{description}

